\documentclass[]{article}

\usepackage{hyperref}
\usepackage{listings}

%opening
\title{Programrendszerek fejlesztés projektmunka}
\author{Gáspár Tamás}

\begin{document}

\maketitle

\tableofcontents

\section{Linkek}

Az alábbiak az alkalmazáshoz tartozó legfontosabb linkek.

\begin{itemize}
	\item \href{https://github.com/Gtomika/prf-project}{A projekt GitHub repozitóriuma}
\end{itemize}

\section{A fejlesztés menete}

Itt részletesebben leírom, hogy egyes commitok mit adtak hozzá, vagy változtattak meg. Minden esetben megadok egy linket is a 
commitra, hogy ez ellenőrizhető legyen. A commitokban ezt a dokumentációt is megváltoztatom, de ezt nem írom itt le. Az olyan commitok nem kerülnek ide, ahol csak a dokumentációt frissítettem. 

\subsection{Alapozás}

\noindent
\textbf{A commit teljes szövege:} \textit{NodeJS és angular hozzáadása. NodeJS kezdeti konfigurálás, Angular alap frontend.}

\bigskip
\noindent
Ez a commit \href{https://github.com/Gtomika/prf-project/commit/170975c76199384bc0c8c9524929d8804e70a56d}{itt} található.
\bigskip

Létrehoztam a projekt struktúráját, majd az Angularos és a NodeJS projekteket. Eldöntöttem, hogy az angular alkalmazást a NodeJS szerver fogja hostolni, és kidolgoztam ennek a módját. 
Letöltöttem a szükséges \textit{npm} csomagokat, és megírtam a szerver vázát. Az angular alkalmazáshoz egyszerű frontendet készítettem, ebbe fognak majd a komponensek kerülni.

\subsection{Felhasználói felület létrehozása}

\noindent
\textbf{A commit teljes szövege:} \textit{Angular frontend fejlesztése}

\bigskip
\noindent
Ez a commit \href{https://github.com/Gtomika/prf-project/commit/53e7acb28718a72afb40870192a166d8839e0d08}{itt} található.
\bigskip

Megalkottam a felhasználói felületet az angularban. A főkomponens tartalmazza a menüsort, és a bejelentkezési állapotot, és ezen belül jelennek meg az egyes komponensek. A hozzáadott komponensek:

\begin{itemize}
	\item Login: bejelentkezés
	\item Register: regisztráció
	\item Home: főoldal
	\item Products: termékek
	\item About: információ
\end{itemize}

\subsection{Authentikáció}

\noindent
\textbf{A commit teljes szövege:} \textit{Felhasználókezelés megvalósítása}

\bigskip
\noindent
Ez a commit \href{https://github.com/Gtomika/prf-project/commit/ac23392a3ebe9a36ad80a358ebb768bba0434b408}{itt} található.
\bigskip

Kliens és szerveroldalon is implementáltam a felhasználókezelést. Ez szerver oldalon azt jelenti, hogy létrehoztam a login, logout és user endpointokat, és az felhasználó adatbázis sémáját. Angular esetén funkcionalitás került a login és register komponensek mögé, és bekerült egy guard, ami ellenőrzi a bejelentkezést.

Még itt elég sok a hiba, a későbbiekben ezen jelentősen javítok. Pl: bejelentkezés ellenőrzése service-el, nem pedig a local storage-al, vagy a loading indicator mutatása amíg a bejelentkezés tart.

\subsection{Felhasználókezelés}

\noindent
\textbf{A commit teljes szövege:} \textit{ Backend: felhasználói adatok frissítése user endpointon. UI: sweetalert, material, felhasználói felületadatmódosításra}

\bigskip
\noindent
Ez a commit \href{https://github.com/Gtomika/prf-project/commit/6794668ee38e4793fa249915b7df71481f39e519}{itt} található.
\bigskip

Szerver oldalon bővítettem a \textit{/user} endpointot, hogy lehessen felhasználói jelszót változtatni, és felhasználót törölni. Kliens oldalon ehhez egy felületet hoztam létre, ahol a bejelentkezett felhasználó módosíthatja a jelszavát vagy törölheti magát.

Szépítettem az oldal kinézetét, a Google material komponensek használatával, és az alap JavaScript alert dialógus helyett egy alert könyvtárra tértem át.

\subsection{Termékkezelés admin felületről}

\noindent
\textbf{A commit teljes szövege:} \textit{ Backend: product endpoint, Frontend: admin felület}

\bigskip
\noindent
Ez a commit \href{https://github.com/Gtomika/prf-project/commit/f592c43be14ac39f53c4aa99b7b46c60c5b0640c}{itt} található.
\bigskip

Szerver oldalon létrehoztam a \textit{/product} endpointot, a termékek módosítására és lekérésére. Ehhez egy admin felületet is létrehoztam. Ezt a felületet csak bejelentkezett admin érheti el. 

Itt lehetőség van új termék létrehozására, meglévő frissítésére, és törlésére. Létrehozás és frissítés úgy történik, hogy fel lehet tölteni egy JSON fájlt, ami a termék adatait tartalmazza. Egy ilyen példa fájlt beraktam a \textit{docs} mappába is, \textit{imaginaryPiece.json} néven.

\subsection{Termék komponens, images endpoint}

\noindent
\textbf{A commit teljes szövege:} \textit{ Backend: images endpoint, Frontend: termék komponens}

\bigskip
\noindent
Ez a commit ... található.
\bigskip

A szerverhez hozzáadtam a \textit{/images} endpointot, amitől el lehet kérni az egyes termékekhez tartozó képeket. Angularban megcsináltam az egy terméket mutató komponenst. Ez a komponens elkéri a szervertől a termék képét, majd megjeleníti.

\section{API pontok}

Itt dokumentálom, hogy milyen API endpointokat lehet használni. Mindegyik endpoint a \textit{/api} mögött van. Ha az URL-ben nincs \textit{/api}, azt az angular fogja lekezelni, nem pedig a szerver.

Például:

\bigskip
\begin{center}
	\textit{http://localhost:3080/api/status}
\end{center}

\subsection{/status}

Ez csak GET-re válaszol, és egyszerűen visszaküld egy kis szöveget. Arra használható, hogy megnézzük hogy működik-e a szerver. Az endpointok a nodeJS mappa endpoints almappájában vannak definiálva, kivéve a \textit{/status}, mert az annyira egyszerű.

\subsection{/login}

Ide csak POST-olni lehet, és meg kell adni a felhasználónevet és jelszót. Ha ezek jók, akkor a bejelentkeztetés megtörténik.

\noindent
Az adatokat JSON-ban várja:

\bigskip
\begin{lstlisting}
{
 "username": "valaki",
 "password": "valaki_jelszava"
}
\end{lstlisting}

\noindent
A választ JSONben küldi:

\bigskip
\begin{lstlisting}
{
 "message": "Pl Sikeres bejelentkezes",
 "isAdmin": "false"
}
\end{lstlisting}

\noindent
Az angular nézi az \textit{isAdmin} értékét, hogy tudja, hogy admin jelentkezett-e be.

\subsection{/logout}

A \textit{/login} párja, ami kijelentkezteti a felhasználót. Csak POST-ot fogad, nem vár semmilyen adatot és akkor sem dob hibát, ha nem volt senki bejelentkezve.

\subsection{/user}

Felhasználókezelő endpoint, ami POST-ot, PUT-ot és DELETE-et támogat.

POST esetén regisztáció történik. JSON-ban meg kell adni a felhasználónevet és jelszót (úgy, mint a \textit{/login} esetén).

PUT esetén a jelszó frissíthető, itt meg kell adni a felhasználónevet és \textbf{jelenlegi} jelszót, majd az új jelszót:

\bigskip
\begin{lstlisting}
{
 "username": "valaki",
 "password": "valaki_jelszava",
 "newPassword": "valaki_uj_jelszava"
}
\end{lstlisting}

Csak akkor lesz változtatás, ha a felhasználó be van jelentkezve, és a jelenlegi jelszava egyezik.

DELETE-tel felhasználó törlés kell. PUT-hoz hasonlóan itt is küldeni kell a felhasználónevet és a jelszót, csak akkor fog végrehajtódni, ha be vagyunk jelentkezve és a jelszó helyes.

\subsection{/product}

Ezen az endpointon kérhetőek le, vagy módosíthatóak a termékek. Mindegyik híváshoz be kell, hogy jelentkezzünk. Azokat hívásokat, amik a termékek adatbázisát megváltoztatják, csak admin hajthatja végre.

\noindent
\textbf{GET}: Ez lekéri az összes terméket, amiket egy JSON fájlban küld vissza.

\noindent
\textbf{POST (admin)}: Új termék hozható létre vele. Várja a hitelesítő adatokat, és a termék adatait, a következő formában:

\bigskip
\begin{lstlisting}
{
 "username": "valaki_admin",
 "password": "valaki_jelszava",
 "name": "termek neve",
 "description": "termek leirasa",
 "price": "termek ara",
 "imgPath": "utvonal a termek kepere"
}
\end{lstlisting}

\bigskip
\noindent
\textbf{PUT (admin)}: POST-hoz hasonló, de itt a terméknek már léteznie kell, és az adatai frissítve lesznek. Pont ugyanolyan formában várja az inputot, mint a POST.

\noindent
\textbf{DELETE (admin)}: Termék törlése. Az inputot a következő formában várja:

\bigskip
\begin{lstlisting}
{
 "username": "valaki_admin",
 "password": "valaki_jelszava",
 "name": "torlendo termek neve",
}
\end{lstlisting}

\subsection{/images}

Ez csak \textit{GET}-et fogad, és lehetőség van vele lekérni a termékekhez tartozó képeket. Az \textit{URL}-ben egy \textit{imageName} nevő paramétert kell megadni, ami megmondja a szervernek, hogy melyik képet kell visszaküldeni. Az angular ezt a termék \textit{imgPath} attribútumából állítja elő. 

Például így lehet a 'Gyalog' nevű termék képét elkérni (debug módban):

\bigskip
\begin{center}
	\textit{http://localhost:3080/api/images?imageName=pawn.png}
\end{center}
\bigskip

Csak bejelentkezett klienstől fogad kéréseket. A válasz maga a kép lesz (\textit{Content-Type: image/jpeg}). 

\section{Bemutató}

Ahogy a követelményekben megvan, lehetőség van a belépni a következő felhasználóval:

\begin{itemize}
	\item Username: szaboz
	\item Jelszó: PRF2021
\end{itemize}

Ez a felhasználó egyben admin is. Lehet regisztrálni persze más felhasználót is, az nem lesz admin.

\end{document}
