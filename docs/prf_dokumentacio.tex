\documentclass[]{article}

\usepackage{hyperref}

%opening
\title{Programrendszerek fejlesztés projektmunka}
\author{Gáspár Tamás}

\begin{document}

\maketitle

\tableofcontents

\section{Linkek}

Az alábbiak az alkalmazáshoz tartozó legfontosabb linkek.

\begin{itemize}
	\item \href{https://github.com/Gtomika/prf-project}{A projekt GitHub repozitóriuma}
\end{itemize}

\section{A fejlesztés menete}

Itt részletesebben leírom, hogy egyes commitok mit adtak hozzá, vagy változtattak meg. Minden esetben megadok egy linket is a 
commitra, hogy ez ellenőrizhető legyen. A commitokban ezt a dokumentációt is megváltoztatom, de ezt nem írom itt le. Az olyan commitok nem kerülnek ide, ahol csak a dokumentációt frissítettem. 

\subsection{Alapozás}

\noindent
\textbf{A commit teljes szövege:} \textit{NodeJS és angular hozzáadása. NodeJS kezdeti konfigurálás, Angular alap frontend.}

\bigskip
\noindent
Ez a commit \href{https://github.com/Gtomika/prf-project/commit/170975c76199384bc0c8c9524929d8804e70a56d}{itt} található.
\bigskip

Létrehoztam a projekt struktúráját, majd az Angularos és a NodeJS projekteket. Eldöntöttem, hogy az angular alkalmazást a NodeJS szerver fogja hostolni, és kidolgoztam ennek a módját. 
Letöltöttem a szükséges \textit{npm} csomagokat, és megírtam a szerver vázát. Az angular alkalmazáshoz egyszerű frontendet készítettem, ebbe fognak majd a komponensek kerülni.

\subsection{Felhasználói felület létrehozása}

\noindent
\textbf{A commit teljes szövege:} \textit{Angular frontend fejlesztése}

\bigskip
\noindent
Ez a commit \href{https://github.com/Gtomika/prf-project/commit/53e7acb28718a72afb40870192a166d8839e0d08}{itt} található.
\bigskip

Megalkottam a felhasználói felületet az angularban. A főkomponens tartalmazza a menüsort, és a bejelentkezési állapotot, és ezen belül jelennek meg az egyes komponensek. A hozzáadott komponensek:

\begin{itemize}
	\item Login: bejelentkezés
	\item Register: regisztráció
	\item Home: főoldal
	\item Products: termékek
	\item About: információ
\end{itemize}

\section{API pontok}

Itt dokumentálom, hogy milyen API endpointokat lehet használni.

\section{Bemutató}

Itt leírom, hogy hogyan kell használni az alkalmazást.

\end{document}
